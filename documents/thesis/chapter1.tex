\chapter{Introdução} \label{chap:intro}

\section*{}

This document covers the work undergone regarding the study, design and implementation of the project titled:

Web System For Creating And Managing Virtual High Performance Computing Environments

throughout the course of two six month semesters.


This chapter describes the motivation behind the choice of this subject, as well as what are the subjects covered throughout the whole document and what new ideas can come from this work.
It is divided in ##XX## parts, starting with a general view on Cloud and Grid computing, since they are two major areas of interest in this project, along with virtualization and other related subjects such as hypervisors. Cloud creation and management toolkits which are available and used throughout the academic community, such as OpenNebula and more recently, OpenStack are also approached.
A possible integration of the project documented and OpenStack is discussed in a section further ahead.

\section{Context - What it is this all about} \label{sec:context}

Leonard Kleinrock (part of the team that developed Arpanet, an early seed for the Internet) said in 1969:

\begin{quote}
  ``As [...] computer networks [...] grow and become sophisticated, we will probably see the spread of `computer utilities' which, like present electric and telephone utilities, will service individual homes and offices around the country.''~\cite{Buyya2009599} 
\end{quote}
	
Confirming Kleinrock's prediction, computing is migrating in a direction where people develop software for an incredible amount of people in order for them to use it as a service, instead of running them on their personal computers. Different providers such as Amazon, Google, IBM and Sun Microsystems are now establishing data centers dedicated to hosting Cloud Computing\footnote{Using multiple server computers via a digital network as if they were a single computer.} applications spread around the world in order to ensure redundancy and reliability in case one of the datacenters fails. 

User requirements for Cloud services are complex and varied, so service providers need to know they can be flexible when delivering those services at the same time they keep the users clear from the infrastructure on which those services stand.

Computing services are available instantly when anyone needs them and the consumers only required to pay the providers when they actually access and use those resources. Consumers no longer have the need to invest in and maintain complex IT infrastructures and software developers are facing new challenges. They must create custom made software that will be used as a service, instead of the traditional practice of installing the software in the users' machines. Some people state this is the era of pervasive computing, where computation and information are available all the time.~\citet{ieees}

Having this in mind, FEUP has started developing a private cloud project at its Informatics center (CICA - Centro de Informática Prof. Correia de Araújo)

\section{Context - A birdseye view on a bigger project} \label{sec:project}

Before going into the areas of interest themselves, it is necessary to provide a general idea of what this project is and the reasoning behind its implementation.
Currently, FEUP's computing infrastructures are only accessed by those who have the technical knowledge to interact with the system. These people are technicians whose area of expertise encompasses outsourcing computing resources to perform computing jobs. If someone from an area unrelated to the computing system wants to perform any operation in it, that someone must contact the said technicians and waste valuable time for both parties cutting through red tape.
Having this in mind, CICA has started developing a project that reduces the ammount of knowledge necessary to perform the said computing operations.
The project aims at simplifying the whole process and to make FEUP's computing infrastructures more accessible to the academic community, without the users having to spend time learning about the technologies and how the system actually works.
This document focuses only on the front-end of the project, the back-end having already been developed by former MIEIC student Nuno Cardoso as part of his Master Thesis.
In order to better understand the full scope of the issue, Figure 1 shows the full system as it should function, through the means of an hypotetic and yet plausible use case scenario:

<<insert figure 1>>

Firstly, a researcher of a specific field of study wants to conduct a more complex operation that involves greater computing efforts than his/her home and work computer. As such, the researcher proceeds to enter the designed system through a web page where he/she can:

\begin{itemize}
	\item Chose a suitable work environment for her computational needs according to a set of predefined parameters;
	\item Create his/her own work environment according to the specifications he/she provides the system with.
\end{itemize}

The system will then automatically create a virtual environment (image containing all the information needed), which will be passed onto the second part of the project where a virtual cluster will be created according to that virtual environment.
Finally a username and password combination should be returned so that the researcher can enter the created environment and perform his/her operations.
	
The fraction of the work covered by this document has as its objectives the automatic creation and management of virtual environments over private high performance computing infrastructures. 





Esta secção descreve a área em que o trabalho se insere, podendo
referir um eventual projecto de que faz parte e apresentar uma breve
descrição da empresa onde o trabalho decorreu.

Lorem ipsum~\citep{kn:Lip08} dolor sit amet, consectetuer adipiscing
elit. 
Sed eget nunc. Phasellus interdum, risus viverra mollis laoreet, felis
justo iaculis ante, eget ornare purus augue non urna. Nam in magna. In a
est. Phasellus a tellus vitae enim vehicula imperdiet. Etiam sit amet
elit. In hac habitasse platea dictumst. Quisque eget turpis vel felis
elementum tempus. Curabitur sit amet tortor id libero dapibus
pretium. Integer mattis eros eu lorem. Duis erat tellus, porttitor
sed, blandit eget, fringilla et, lacus. Phasellus tristique nibh nec
orci. Mauris sed leo. Suspendisse fringilla tempor dolor. Donec sapien
enim, congue in, porta et, sollicitudin in, quam. Curabitur semper,
mauris ut vestibulum eleifend, diam ipsum tincidunt quam, et
vestibulum velit mauris ut risus. 

Sed eget libero. Nulla facilisi. Proin eget tortor. Morbi
gravida. Donec arcu risus, blandit a, rutrum at, ornare ut,
nisl. Etiam consectetuer tortor eu odio. Etiam blandit molestie
ligula. Nulla facilisi. Nam a augue non justo laoreet hendrerit. Nam
aliquam, purus eu ultricies dictum, urna purus posuere neque, vel
tempus tellus enim a arcu. 

\section{Projecto} \label{sec:proj}

Na continuação da secção anterior, e apenas no caso de ser um Projecto
e não uma Dissertação, esta secção apresenta resumidamente o projecto.

Nulla nec eros et pede vehicula aliquam. Aenean sodales pede vel
ante. Fusce sollicitudin sodales lacus. Maecenas justo mauris,
adipiscing vitae, ornare quis, convallis nec, eros. Etiam laoreet
venenatis ipsum. In tellus odio, eleifend ac, ultrices vel, lobortis
sed, nibh. Fusce nunc augue, dictum non, pulvinar sed, consectetuer
eu, ipsum. Vivamus nec pede. Pellentesque pulvinar fringilla dolor. In
sit amet pede. Proin orci justo, semper vel, vulputate quis, convallis
ac, nulla. Nulla at justo. Mauris feugiat dolor. Etiam posuere
fermentum eros. Morbi nisl ipsum, tempus id, ornare quis, mattis id,
dolor. Aenean molestie metus suscipit dolor. Aliquam id lectus sed
nisl lobortis rhoncus. Curabitur vitae diam sed sem aliquet
tempus. Sed scelerisque nisi nec sem. 

\section{Motivação e Objectivos} \label{sec:goals}

Apresenta a motivação e enumera os objectivos do trabalho terminando
com um resumo das metodologias para a prossecução dos objectivos.

Lorem ipsum dolor sit amet, consectetuer adipiscing elit. Morbi sit
amet nibh. Fusce faucibus, enim vel ultrices ornare, est mauris
ultricies velit, vitae consequat sem erat vel nunc. Nam libero eros,
mattis eget, sagittis nec, imperdiet at, sapien. Aliquam lacus. Aenean
adipiscing nibh in orci. Aliquam vestibulum, elit at fringilla
dignissim, metus diam lobortis urna, a laoreet nunc odio ac ipsum. Sed
at urna. Integer vehicula fringilla augue. Nulla lacus eros, rhoncus
sit amet, posuere ut, vehicula ac, nibh. Ut eleifend, eros eu placerat
vehicula, justo turpis blandit dolor, eu tincidunt felis risus at
ante. Aenean suscipit nisl eget eros. Ut laoreet libero eget
enim. Cras tempus pellentesque felis. Vestibulum vitae erat ac nibh
posuere eleifend. 

Integer nec quam. Sed fermentum. Nunc vitae leo. Etiam sit amet
quam. Nunc vestibulum massa in mauris. Duis eget nulla. Fusce
ultricies arcu eu nibh volutpat feugiat. Maecenas urna pede, commodo
quis, porta eu, bibendum elementum, pede. Sed eros massa, molestie
eget, mattis non, rutrum ac, magna. Duis dui. Maecenas eget tortor ut
dolor semper mattis. Maecenas auctor, tellus et ultricies tempor, elit
est placerat lacus, in posuere mauris lorem et arcu. 

\section{Estrutura da Dissertação} \label{sec:struct}

Para além da introdução, esta dissertação contém mais x capítulos.
No capítulo~\ref{chap:sota}, é descrito o estado da arte e são
apresentados trabalhos relacionados. 
No capítulo~\ref{chap:chap3}, ipsum dolor sit amet, consectetuer
adipiscing elit.
No capítulo~\ref{chap:chap4} praesent sit amet sem. 
No capítulo~\ref{chap:concl}  posuere, ante non tristique
consectetuer, dui elit scelerisque augue, eu vehicula nibh nisi ac
est. 
