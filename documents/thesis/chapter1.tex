\chapter{Introdução} \label{chap:intro}

\section*{}
This chapter describes the motivation behind the choice of this subject, as well as what are the subjects covered throughout the whole document and what new ideas can come from this work.
It is divided in ##XX## parts, starting with a general view on some of the major areas of interes in this project, such as what is Cloud and Grid computing, what is virtualization and why it is discussed in great extents, what are hypervisors and how they matter and a brief introduction on what will be talked about further on.

\section{Context - The birdseye view on a bigger project} \label{sec:context}

Before going into the areas of interest themselves, it is necessary to provide a general idea on what this project lies upon. 
This project is part of a bigger project, currently under development at CICA - Centro de Informática Correia de Araújo. This project is the second part of a two-part work, where the first part was developed last semester by MIEIC student Nuno Cardoso, as part of his Master Thesis.
The idea behind this project is to improve on FEUP's Cloud Computing system, through the completion of several goals. 
In order to better understand the full scope of the issue, Figure 1 shows the full system as it should function, through the means of an hypotetic and yet plausible use case scenario:

<<insert figure 1>>

Firstly, a researcher of a specific field of study wants to conduct a more complex operation that involves greater computing efforts than his/her home and work computer. As such, the researcher proceeds to enter the designed system through a web page where he/she can:

\begin{itemize}
	\item Chose a suitable work environment for her computational needs according to a set of predefined parameters;
	\item Create his/her own work environment according to the specifications he/she provides the system with.
\end{itemize}

The system will then automatically create a virtual environment (image containing all the information needed), which will be passed onto the second part of the project where a virtual cluster will be created according to that virtual environment.
	
The fraction of the work covered by this document has as its objectives the automatic creation and management of virtual environments over private high performance computing infrastructures. 





Esta secção descreve a área em que o trabalho se insere, podendo
referir um eventual projecto de que faz parte e apresentar uma breve
descrição da empresa onde o trabalho decorreu.

Lorem ipsum~\citep{kn:Lip08} dolor sit amet, consectetuer adipiscing
elit. 
Sed eget nunc. Phasellus interdum, risus viverra mollis laoreet, felis
justo iaculis ante, eget ornare purus augue non urna. Nam in magna. In a
est. Phasellus a tellus vitae enim vehicula imperdiet. Etiam sit amet
elit. In hac habitasse platea dictumst. Quisque eget turpis vel felis
elementum tempus. Curabitur sit amet tortor id libero dapibus
pretium. Integer mattis eros eu lorem. Duis erat tellus, porttitor
sed, blandit eget, fringilla et, lacus. Phasellus tristique nibh nec
orci. Mauris sed leo. Suspendisse fringilla tempor dolor. Donec sapien
enim, congue in, porta et, sollicitudin in, quam. Curabitur semper,
mauris ut vestibulum eleifend, diam ipsum tincidunt quam, et
vestibulum velit mauris ut risus. 

Sed eget libero. Nulla facilisi. Proin eget tortor. Morbi
gravida. Donec arcu risus, blandit a, rutrum at, ornare ut,
nisl. Etiam consectetuer tortor eu odio. Etiam blandit molestie
ligula. Nulla facilisi. Nam a augue non justo laoreet hendrerit. Nam
aliquam, purus eu ultricies dictum, urna purus posuere neque, vel
tempus tellus enim a arcu. 

\section{Projecto} \label{sec:proj}

Na continuação da secção anterior, e apenas no caso de ser um Projecto
e não uma Dissertação, esta secção apresenta resumidamente o projecto.

Nulla nec eros et pede vehicula aliquam. Aenean sodales pede vel
ante. Fusce sollicitudin sodales lacus. Maecenas justo mauris,
adipiscing vitae, ornare quis, convallis nec, eros. Etiam laoreet
venenatis ipsum. In tellus odio, eleifend ac, ultrices vel, lobortis
sed, nibh. Fusce nunc augue, dictum non, pulvinar sed, consectetuer
eu, ipsum. Vivamus nec pede. Pellentesque pulvinar fringilla dolor. In
sit amet pede. Proin orci justo, semper vel, vulputate quis, convallis
ac, nulla. Nulla at justo. Mauris feugiat dolor. Etiam posuere
fermentum eros. Morbi nisl ipsum, tempus id, ornare quis, mattis id,
dolor. Aenean molestie metus suscipit dolor. Aliquam id lectus sed
nisl lobortis rhoncus. Curabitur vitae diam sed sem aliquet
tempus. Sed scelerisque nisi nec sem. 

\section{Motivação e Objectivos} \label{sec:goals}

Apresenta a motivação e enumera os objectivos do trabalho terminando
com um resumo das metodologias para a prossecução dos objectivos.

Lorem ipsum dolor sit amet, consectetuer adipiscing elit. Morbi sit
amet nibh. Fusce faucibus, enim vel ultrices ornare, est mauris
ultricies velit, vitae consequat sem erat vel nunc. Nam libero eros,
mattis eget, sagittis nec, imperdiet at, sapien. Aliquam lacus. Aenean
adipiscing nibh in orci. Aliquam vestibulum, elit at fringilla
dignissim, metus diam lobortis urna, a laoreet nunc odio ac ipsum. Sed
at urna. Integer vehicula fringilla augue. Nulla lacus eros, rhoncus
sit amet, posuere ut, vehicula ac, nibh. Ut eleifend, eros eu placerat
vehicula, justo turpis blandit dolor, eu tincidunt felis risus at
ante. Aenean suscipit nisl eget eros. Ut laoreet libero eget
enim. Cras tempus pellentesque felis. Vestibulum vitae erat ac nibh
posuere eleifend. 

Integer nec quam. Sed fermentum. Nunc vitae leo. Etiam sit amet
quam. Nunc vestibulum massa in mauris. Duis eget nulla. Fusce
ultricies arcu eu nibh volutpat feugiat. Maecenas urna pede, commodo
quis, porta eu, bibendum elementum, pede. Sed eros massa, molestie
eget, mattis non, rutrum ac, magna. Duis dui. Maecenas eget tortor ut
dolor semper mattis. Maecenas auctor, tellus et ultricies tempor, elit
est placerat lacus, in posuere mauris lorem et arcu. 

\section{Estrutura da Dissertação} \label{sec:struct}

Para além da introdução, esta dissertação contém mais x capítulos.
No capítulo~\ref{chap:sota}, é descrito o estado da arte e são
apresentados trabalhos relacionados. 
No capítulo~\ref{chap:chap3}, ipsum dolor sit amet, consectetuer
adipiscing elit.
No capítulo~\ref{chap:chap4} praesent sit amet sem. 
No capítulo~\ref{chap:concl}  posuere, ante non tristique
consectetuer, dui elit scelerisque augue, eu vehicula nibh nisi ac
est. 
