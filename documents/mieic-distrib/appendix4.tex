\chapter{vmbuilder script} \label{chap:ap4}

Script used to create VM images dynamically.

This creates a \texttt{KVM Ubuntu} image, version \textit{Precise Pangolin} (\texttt{--suite=precise}), suitable for virtual environments (\texttt{--flavour=virtual}), for 32 bit machine (\texttt{--arch=i386}), using the German \textit{Ubuntu} mirrors to get the packages (\texttt{--mirror}).

The section \texttt{-o --libvirt=qemu:///system} tells the system to register the newly created with the system's virtual machine manager.

In this case, a file named \texttt{vmbuilder.partition} was used to define the disk partitioning. The section \texttt{--templates=templates} points to the folder where \texttt{vmbuilder} should use the templates to build the image. 

After this, we have the definition of the user, the name to be used and the password.

\texttt{--addpkg} tells \texttt{vmbuilder} to install the security updates, \texttt{vim} and \textit{acpid} (used for functions such as closing a laptop lid, pressing the power button, etc).

\texttt{--firstboot} refers to the script which will be run as soon as the VM image is booted. In this case, the script \texttt{boot.sh} tells the newly created machine to read the information from the repositories in order to find new packages (since this is a first boot, all of them will be new) and installs \texttt{ssh}.

Finally, \texttt{--mem=256} specifies the total RAM and \textit{--hostname}, defines the machine's hostname.
\begin{verbatim}
#!/bin/bash
sudo vmbuilder kvm ubuntu --suite=precise --flavour=virtual \
	--arch=i386 --mirror=http://de.archive.ubuntu.com/ubuntu \
	-o --libvirt=qemu:///system --ip=192.168.0.101 \
	--part=vmbuilder.partition --templates=templates \
	--user=user --name=Administrator --pass=password \
	--addpkg=vim-nox --addpkg=unattended-upgrades \
	--addpkg=acpid --firstboot=/home/pedro/Desktop/scripts/boot.sh \
	--mem=256 --hostname=CHEM_IMG
\end{verbatim}
