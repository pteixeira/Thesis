\chapter{Introduction} \label{chap:intro}


High performance computing describes the ability of using parallel processing in order to perform advanced application programs with a great deal of efficiency, reliability and quickness. \cite{hpclinux}

Current Grid Computing infrastructures are generally not very flexible when it comes to the users' needs. As such, whenever it is required, the user must adapt its code to the infrastructures specifications.

On the other hand, Cloud Computing is associated with an extreme flexibility allowing the infrastructure to adapt itself to the users' requirements. Another aspect present in Cloud Computing but non-existent in Grid Computing is the Quality of Service factor, where a user can submit a job according to a certain cost or deadline.

Furthermore, there is also the elasticity component, something that is not available in Grid Computing technologies but is inherent to Cloud Computing, and is one of its flagships that may be able to cross over to grid infrastructures.

In this document is presented a study that analyzes two community projects on creating and managing private Cloud infrastructures (along with some of the technologies that support these projects) and if and how they can be implemented in FEUP - Faculty of Engineering of the University of Porto.

This first chapter introduces a brief technological and situational context, as well as the motivation behind the choice of this subject and the objectives set. The document's structure is also shown.

\section{Context} \label{sec:context}

Leonard Kleinrock (part of the team that developed Arpanet, an early seed for the Internet) said in 1969:

\begin{quote}
  ``As [...] computer networks [...] grow and become sophisticated, we will probably see the spread of `computer utilities' which, like present electric and telephone utilities, will service individual homes and offices around the country.''~\cite{Buyya2009599} 
\end{quote}
	
Confirming Kleinrock's prediction, computing is migrating in a direction where people develop software for an incredible amount of people so it can be used as a service, instead of running said software on their personal computers. Different providers such as Amazon, Google, IBM and Sun Microsystems are now establishing data centers dedicated to hosting Cloud Computing\footnote{Using multiple server computers via a digital network as if they were a single computer.} applications spread around the world in order to ensure redundancy and reliability in case one of the datacenters fails. 

User requirements for Cloud services are complex and varied, so service providers need to know they can be flexible when delivering those services at the same time they keep the users clear from the infrastructure on which those services stand.

Computing services are available instantly when anyone needs them and the consumers only required to pay the providers when they actually access and use those resources. Consumers no longer have the need to invest in and maintain complex IT infrastructures and software developers are facing new challenges. They must create custom made software that will be used as a service, instead of the traditional practice of installing the software in the users' machines. Some people state this is the era of pervasive computing, where computation and information are available all the time.~\citet{ieees}

Having this in mind, FEUP has started developing a private cloud project at its Informatics center (CICA - Centro de Informática Prof. Correia de Araújo). As it will be discussed in greater detail in Chapter 3 - \nameref{chap:chap3} - this document reports the work realized on the front-end of the private cloud project.

\section{Motivation and Objectives} \label{sec:motivation}

Some computing infra-structures, namely Grids\footnote{Distributed systems that are loosely coupled, heterogeneous and geographically dispersed and act together to perform very large tasks.} and Clusters\footnote{Group of linked computers working closely together as if they were a single machine.}, can be rather inflexible when compared to Clouds, as the latter are supposed to allow the user to take advantage of a myriad of services, and not just computing power.~\cite{brighthub}
	
However, as powerful as these infra-structures can be, they can be deemed useless if people who need to work with them, cannot do it because they have no knowledge of the technologies. As such, this type of issue causes a lack of growth in the use of FEUP's computing system.

This project aims at increasing the usability of the current computing system that exists at FEUP and with this, increase its usage and stop the lack of growth. In order to achieve this goal, it was purposed that a web portal would be developed which would simplify the access to the system. This portal would have a list of software packets and Linux distributions that the user could choose from and create an ISO image which would run the investigator's computing job.

\section{Dissertation Structure} \label{sec:structure}

This Dissertation is structured as follows:\\
\textbf{Chapter \ref{chap:intro}: ``\nameref{chap:intro}'' } --- This chapter.\\
\textbf{Chapter \ref{chap:sota}: ``\nameref{chap:sota}'' } --- Bibliographic review on some of the most relevant areas for this project and on the technologies that could (and some will) be used.\\
\textbf{Chapter \ref{chap:chap3}: ``\nameref{chap:chap3}'' } --- Exposes the ``problem'' that originated this dissertation, as well as the relevance of the solution proposed.\\
\textbf{Chapter \ref{chap:chap4}: ``\nameref{chap:chap4}'' } --- Reviews the current state of the implementation. FIXME\\
\textbf{Chapter \ref{chap:concl}: ``\nameref{chap:concl}'' } --- Reviews the project, drawing conclusions on what was implemented and what remains to be done, with reference to Chapter~\ref{chap:chap4}. It provides a summary of the contributions and the future work and how it can be used for whoever wishes to work with these technologies in these environments.\\


