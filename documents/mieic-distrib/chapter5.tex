\chapter{Conclusion} \label{chap:concl}

The problem described in Chapter~\ref{chap:chap3} has been solved as described in Chapter~\ref{chap:chap4}

\section{Conclusions}\label{sec:conclusions}

During the realization of this project a few conclusions were drawn.

First of all, \textit{OpenStack} is a major competitor in the cloud computing scene. If it continues with the same pace it had (and maintained) since its first release, it will overthrow its competitors. It is extremely powerful and has the community and industry backup.

Secondly, the generation of VM images on-the-fly is only worth it when there are few to no usable VMs present in the system. As time passes, almost every scenario will be covered and the need for VM image creation will decrease. It may be profitable to setup a VM image repository so that others can benefit from the VM images created by this project, but further research is needed in this subject.

In addition, the \textit{Django} framework came to be a powerful tool in this project. Its MVC architecture simplified the development process and \textit{Python} is an easy to pickup programming language with a very powerful community behind it.

\section{Future Work}\label{sec:future-work}

One of the main improvements that can be done is the full integration with \textit{OpenStack}'s dashboard, instead of relying on a middleman (the web application).

Another improvement would be the possibility of adding new packages to the VM images configuration by user input, instead of choosing them on a fixed list, since this limits what the user can choose from. Removing this limitation can potentially allow the project to scale outside of FEUP's range and open the possibility of deployment on other facilities.
