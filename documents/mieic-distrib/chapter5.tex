\chapter{Conclusion} \label{chap:concl}

The problem described in Chapter~\ref{chap:chap3} has been solved as described in Chapter~\ref{chap:chap4}

\section{Conclusions}\label{sec:conclusions}

During the realization of this project a few conclusions were drawn.

First of all, \textit{OpenStack} is a major competitor in the cloud computing scene. If it continues with the same pace it had (and maintained) since its first release, it will overthrow its competitors. It is extremely powerful and has both the community and industry backup.

Secondly, the creation of VM images on-the-fly is only worth it when there are few to no usable VMs present in the system. As time passes, almost every scenario will be covered and the need for VM image creation will decrease. It may be profitable to setup a VM image repository so that others can benefit from the VM images created by this project, but further research is needed in this subject.

In addition, the \textit{Django} framework came to be a powerful tool in this project.

Its MVC architecture simplified the development process and \textit{Python} is an easy to pickup programming language with a very strong community behind it. The downside was that even though \textit{Python} is relatively simple, \textit{Django} can become troublesome in some areas. If someone is not used to this kind of \textit{frameworks}, they will have a somewhat slow learning process. 

It is easy to be stuck in the same step for quite some time and not understanding why something is not working, as even though the MVC architecture helps in separating the user interface from the rest of the code, understanding the connections between the views and the controllers can be hard. Most of the issues encountered revolved around the same problem: defining which regular expressions to use for the URL recognition (\textit{Django} uses regular expressions so automatically identify which view to use according to the URL in the browser). Luckily \textit{Django} (similarly to what was described in \nameref{subsec:contextualization} with \textit{cloud-init}) has its own IRC channel in the Freenode IRC network~\footnote{irc.freenode.net} and the users in there were able to solve most of the issues encountered.

\textit{Django} and \textit{Python}'s full potential is only unleashed after the editor the developers use is fully optimized in terms of \textit{plugins}. Due to the great number of functions needed to be used (most of them follow the same schema), use of a template language to code the views (which leads once more to a great repetition in the code process), having the appropriate \textit{plugins} to reduce the amount this repetition can reduce the coding time by a huge amount. \textit{Plugins} like \texttt{Snippets} for \textit{gedit} (the text editor used for this project) improved the coding time after they were installed.

\section{Future Work}\label{sec:future-work}

One of the main improvements that can be done is the full integration with \textit{OpenStack}'s dashboard, instead of relying on a middleman (the web application).

Another improvement would be the possibility of adding new packages to the VM images configuration by user input, instead of choosing them on a fixed list, since this limits what the user can choose from. Removing this limitation can potentially allow the project to scale outside of FEUP's range and open the possibility of deployment on other facilities. 

The direct comparison with \textit{OpenNebula} would be a great contribution as well. Comparing VM creation and contextualization time, even comparing the development process by using \textit{Ruby on Rails} would lead to discover which process is more appropriate and benefitial.
