\chapter*{Abstract}

Current Grid computing infrastructures are generally not very flexible when it comes to the users' needs. As such, whenever it is required, the user must adapt its code to the infrastructures specifications.

On the other hand, Cloud Computing is associated with an extreme flexibility allowing the infrastructure to adapt itself to the users' requirements. Another aspect present in Cloud Computing but non-existent in Grid Computing is the Quality of Service factor, where a user can submit a job according to a certain cost or deadline.

FEUP --- Faculty of Engineering of Port University --- has started developing a private cloud project at its Informatics center (CICA --- Informatics Center Prof. Correia de Araújo) --- in which a user can create custom Virtual Machine images on-the-fly and have the system automatically provision the required resources to run the submitted job.

\textit{OpenStack} and \textit{OpenNebula} are competing cloud management platforms, both with their own methods of dealing with Virtual Machine images. 

In this report both cloud management platforms are reviewed and a choice is made as to which one to use for integrating with the developed web system which serves as a portal to FEUP's private cloud project. The technologies that support these platforms are discussed so that the environment on which they are inserted can be identified.

The work and research involved in creating a web system based on \textit{Python} and \textit{Django} that is capable of creating Virtual Machine images according to the users requisites, as well as capable of managing those said VM images is also documented. 

The integration with \textit{OpenStack} is presented and the advantages the implemented web system has over the mentioned cloud middleware tool are documented.


\chapter*{Resumo}

As infraestruturas actuais de computação em Grelha geralmente não são muito flexíveis no que diz respeito às necessidades dos utilizadores. Assim sendo e sempre que é necessário, é o utilizador que tem de adaptar o código às especificações das infraestruturas.

Por outro lado, a computação em Nuvem é associada a uma flexibilidade extrema, permitindo assim que seja a estrutura a adaptar-se aos requisitos do utilizador. Outro aspecto presente neste tipo de computação, mas que é totalmente ausente na computação em Grelha, é o factor Qualidade de Serviço, em que um utilizador pode submeter um trabalho de acordo com um determinado custo ou um determinado prazo.

A FEUP --- Faculdade de Engenharia da Universidade do Porto --- começou a desenvolver o seu próprio projecto de nuvem privada no seu centro de informática (CICA --- Centro de Informática Prof. Correia de Araújo) em que um utilizador pode criar as suas imagens de máquinas virtuais \textit{on-the-fly} e ser o sistema a provisionar os recursos necessários para correr o trabalho de computação desejado.

\textit{OpenStack} e \textit{OpenNebula} são duas plataformas de gestão de nuvens, competidoras no mercado, sendo que ambas possuem os seus próprios meios de lidar com imagens de máquinas virtuais.

Neste documento as duas plataformas de gestão de nuvens são revistas e uma delas é escolhida para ser usada no projecto de nuvem privada da FEUP. As tecnologias que servem de suporte a estas plataformas são também abordadas, para que o leitor consiga sentir-se contextualizado.

O trabalho e pesquisa envolvido na criação de um sistema web baseado em \textit{Python} e \textit{Django} que é capaz de criar imagens de máquinas virtuais de acordo com os requisitos do utilizador, bem como capaz de gerir essas imagens é também documentado.

A integração com \textit{OpenStack} é apresentada e as vantagens que o sistema web implementado tem sobre a ferramenta de middleware para clouds são documentadas.
