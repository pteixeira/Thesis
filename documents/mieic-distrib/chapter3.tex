\chapter{Problem Statement} \label{chap:chap3}

In this chapter the problem will be described and justified, using references from the bibliographic study presented in Chapter~\ref{chap:sota}.

The problem statement includes the project's requirements specification, which in turn includes the stakeholder identification. 

A design for the solution is presented, including the UML (Unified Modeling Language) diagram, so that the thought process behind the design is more clear. Use cases are also described so that a clear view on what the system does can be obtained. 

The details for web system implementation are described in the next chapter (Chapter \ref{chap:chap4} --~\nameref{chap:chap4}).

\section{Problem Description}

Currently, FEUP's computing infrastructures are only accessed by those who have the technical knowledge to interact with the system. These people are technicians whose area of expertise encompasses outsourcing computing resources to perform computing jobs. 

If someone from an area unrelated to the computing system wants to perform any operation in it, that someone must contact the said technicians and waste valuable time for both parties cutting through red tape.

Having this in mind, CICA has started developing a project that reduces the amount of knowledge necessary to perform the said computing operations.

This document focuses only on the front-end of the project, the back-end having already been developed by former MIEIC student Nuno Cardoso as part of his Master Thesis. CICA's project is described in greater detail in the following section.

\section{The project} \label{sec:project}

In this section the project itself is described in a greater detail, along with the implemented architectural solution.

The use case scenarios are presented, as well as the UML diagram which shows the entities involved in the web application. As it is customary when designing a software application, a requirements elicitation process was held and the results are also shown in this section.

\subsection{The big picture}\label{subsec:bigpicture}

As it was mentioned earlier in this chapter and in the document, the work depicted in this dissertation is part of a bigger project currently being worked on at CICA (FEUP's Informatics Center).

This project aims at simplifying the process of submitting a computing job into FEUP's computing infrastructures, thus making them more accessible to the academic community, without the users having to spend time learning about the technologies and how the system actually works.

In order to better understand the full extention of the issue, Figure~\ref{fig:big_picture} shows the whole system as it should function, through the means of an hypothetic and yet plausible use case scenario:

\begin{figure}[h!]
  \begin{center}
    \leavevmode
    \fbox{\includegraphics[width=\linewidth]{big_picture.png}}
    \caption{CICA's full computing project.}
    \label{fig:big_picture}
  \end{center}
\end{figure}

Firstly, a researcher of a specific field of study wants to conduct a more complex operation that involves greater computing efforts than his/her home and/or work computer. As such, the researcher proceeds to access the designed system through a web page where he/she can:

\begin{itemize}
	\item Choose a suitable work environment for his/her computational needs according to a set of available options;
	\item Create his/her own work environment according to the specifications he/she provides the system with, namely which programs are needed for the computing job.
\end{itemize}

The system will then automatically create a VM image which will be the environment where the computing job will be run. This VM image will be passed onto the back-end of the project where a virtual cluster will be created according to that virtual environment.

Finally a username and password combination should be returned so that the researcher can enter the created environment and perform his/her operations. 

\subsection{The stakeholders}\label{subsec:stakeholders}

The following stakeholders were identified for this project:

\begin{itemize}
\item FEUP researchers;
\item Web system administrator;
\end{itemize} 

These are the only people or groups of people which will interact with the server.

\subsection{The objectives}\label{subsec:objectives}

Having the project outlined, the following objectives were set:

\begin{enumerate}
\item The web system must be able to create VM images;
\item The VM image creation must be dynamic, i.e. the images must be created according to the users specifications, which are inputted via the web system --- contextualization;
\item The web system must help the users regarding which VM image might be more suitable for their needs by providing the appropriate means for that decision;
\item The web system must provide a way for managing the VM images which were registered in it. This management can be performed by either the users (with limitations) or the system administrator;
\item All the above features must be made transparent for the user (the user does not need to know how things work, as long as they do).
\end{enumerate}

These objectives can be translated into system requirements, which are stated in the next section of this chapter.

\subsection{Requirements specification}\label{subsec:requirements}

As it was mentioned, a software application has certain requirements which must be met so that it complies with the system stakeholders' necessities.

As such, interviews were held with some of these stakeholders in order to understand how the system should behave and what functionalities it should offer.

These requirements can be split in two categories:

\begin{enumerate}
\item Functional requirements --- what the system should do and what functions it should perform~\cite{http://dictionary.reference.com/browse/functional+requirements};
\item Non-functional requirements --- Constraints or restrictions that must be considered when designing the solution~\cite{http://www.requirementsauthority.com/functional-and-non-functional.html}.
\end{enumerate}

\subsubsection{Functional requirements}\label{subsubsec:funct-reqs}

The identified functional requirements are:

\begin{itemize}
\item The system must allow the creation of VM images;
\item This creation must be dynamic -- the user must be able to choose what software is to be installed;
\item The user must be presented with tools which allow him/her to better choose which VM image is better suited for his/her needs or if a new VM image must be created;
\item The VM images must be manageable -- they need to have the option to be modified and/or deleted, as well as located within the web system;
\item The VM image specification within the system must have a certain degree of granularity, so that the previous requirement is able to be met;
\item The system must allow the separation of users -- login system, with different permissions for different users;
\end{itemize}

\subsubsection{Non-functional requirements}\label{subsubsec:nonfunct-reqs}

The identified non-functional requirements are:

\begin{itemize}
\item The system must be intuitive;
\item The system must be as simply designed as possible;
\item Users without specific technical knowledge in computing technologies must be able to use the system.
\end{itemize}

\section{The solution}\label{sec:solution}

In this section the solution is documented and the thought process behind it is justified. An UML diagram is included for better understanding on how the system works. 

\subsection{UML diagram}\label{subsec:uml-diag}

Once the requirements and the stakeholders were defined, a diagram depicting the relations between the different elements in the system was produced and it is shown in figure~\ref{fig:uml-diag}.

\begin{figure}[h]
  \begin{center} 
    \leavevmode 
    \includegraphics[width=\textwidth]{class_diagram}
    \caption{Entities and their relationship in the system.} 
    \label{fig:uml-diag} 
  \end{center}
\end{figure}

As it can be observed, the following entities are portrayed:

\begin{itemize}
\item Image -- represents the VM images that are created and registered in the system;
\item User -- represents the users that are registered in the system;
\item User Tasks -- represents which task the user has running in the system. This refers to the VM image creation process;
\end{itemize}

The following relationships between the aforementioned entities can be observed:

\begin{itemize}
\item A user can create several VM images, but one image can only have one owner (this, however, does not mean that an image cannot be used by several users);
\item An image can have several ``tags'' and one ``tag'' can by used by several images;
\item A user can have one task running in the system, specifically one creation of a VM image (this is explained in greater detail in the next chapter, \nameref{chap:chap4});
\item A user can use a VM image, this being logged in the system for statistical purposes;
\item The ``tag'' search is also logged for statistical purposes and to help users select a suitable VM image for their needs.
\end{itemize}

\subsection{Use Cases}\label{sec:use-cases}

Alongside the delineation of the entities, relationships and requirements, there is also the need to delineate the use cases for the web system. These describe what actions can be taken inside the web system, either by normal users or the system administrator.

Use case diagrams are used as visual aids for their textual representation and some screenshots are provided. 

\subsubsection{UC1 -- Login into the web system}\label{uc1}

\textbf{Actors}:

\begin{itemize}
\item Researcher/Administrator.
\end{itemize}

\textbf{Brief description}:

The researcher wishes to log into the web system.

\textbf{Basic flow}:

\begin{itemize}
\item The web system prompts the researcher/administrator for his/her username and password combination (provided beforehand by the system administrator in order to maintain control over who accesses the web system);
\item The researcher inputs his/her username and password in the boxes provided;
\item The web system validates the entered username and password and logs the researcher/administrator into the system.
\end{itemize}

\textbf{Alternative flows}:

If an incorrect username and password combination is supplied, the page will simply reload. The user must enter a valid username and password combination to proceed.

\textbf{Pre-conditions}:

None.

\textbf{Post-Conditions}:

If the use case is successful, the user is logged into the system. If the user is a researcher, no special privileges will be gained. If the user is an administrator, ``administrator'' status will be granted and he/she will gain administration privileges over the system.

If the use case is unsuccessful, no changes are made to the system state.

\subsubsection{UC2 -- Perform management operations}\label{uc2}

\textbf{Actors}:

\begin{itemize}
\item System administrator.
\end{itemize}

\textbf{Brief description}:

The system administrator wants to perform management operations in the system.

\textbf{Basic flow}:

\begin{itemize}
\item In order to perform management operations, the system administrator must login into the system administration panel, which has a different URL from the web system the researchers use;
\item The system administrator must perform the login action as described in the first use case (\nameref{uc1}) using his/her credentials, which are the same for both parts of the system;
\item Inside the administration section, the administrator is able to perform several tasks:
\begin{itemize}
\item Manage the users who can access the web system. This includes adding and deleting users; editing users' details and promoting users to administrators.
\item Manage the database of the system. The administrator can: add VM images to the database; edit VM images' details; and edit or add everything that can be registered by researchers (this may be helpful for testing purposes).
\end{itemize}
\item After the changes are saved, the state of the system is updated.
\end{itemize}

\textbf{Alternative flows}:

None.

\textbf{Pre-conditions}:

User must be logged in as administrator.

\textbf{Post-Conditions}:

The system state will change to whichever operations the system administrator chose to perform.

\subsubsection{UC3 -- Create a new VM image}\label{uc3}

\textbf{Actors}:

\begin{itemize}
\item Researcher.
\end{itemize}

\textbf{Brief description}:

A researcher wishes to create a new VM image.

\textbf{Basic flow}:

\begin{itemize}
\item In order to create a new VM image, the researcher must be logged into the system;
\item After a successful login, the researcher must select the appropriate option from the list provided by the web system (``Create a new VM image'');
\item After being redirected to the VM creation page, the researcher must fill in the details for the newly created VM image:
	\begin{itemize}
	\item Name of the VM image;
	\item Tags to be used for the VM image. These should describe the VM image with as much detail as possible so that other users can find it if they need a similar environment;
	\item Mark the image as ``Public'' or not. This will affect whether the VM image will be available for use by other researchers;
	\item Select which packages they wish to install in the new VM.
	\end{itemize}
\item After filling the required details, the researcher must click on the ``Create'' button;
\item The researcher will then be redirected to the ``Post VM image creation'' page which will be refreshed every few minutes so that the VM creation process status can be updated and displayed to the researcher. After the VM is created, the researcher will be prompted on whether the system should launch the VM;
\end{itemize}

\textbf{Alternative flows}:

The VM image name is already in use. In this case the VM image creation will fail and the page will be refreshed so that the VM image details can be filled out one more time.

\textbf{Pre-conditions}:

User must be logged in.

\textbf{Post-Conditions}:

If the use case is successful, a new VM image will be inserted into the system. IF the use case fails, no changes will occur.

If the researcher chooses to launch the VM image, he/she will go into the use case~\ref{uc4} ---~\nameref{uc4}.

\subsubsection{UC4 -- Launch a VM}\label{uc4}

\textbf{Actors}:

\begin{itemize}
\item Researcher.
\end{itemize}

\textbf{Brief description}:

A researcher wishes to launch a VM.

\textbf{Basic flow}:

\begin{itemize}
\item In order to launch a VM, the user must be logged in;
\item After a successful login, the researcher has three ways of launching a VM image, depending on whether he/she was the creator of the VM image:
	\begin{itemize}
	\item If he/she created the VM image that he/she wishes to launch, the VM image can be found in either by:
		\begin{itemize} 
		\item The ``User details'' page, which can be accessed anywhere in the system by simply clicking on the name that appears on the top-right corner in every page. In the ``User details'' page the researcher must click the name of the VM image and afterwards click the ``Launch VM'' button;
		\item Clicking the link ``Launch an already existing VM image'' located in the system's index page, which will show the user's created VM images, as well as the VM images in the system which are marked as ``Public''. The researcher will then have to click the VM image's name and click the ``Launch VM'' button displayed in the VM image details's page.
		\end{itemize}
	\item If he/she did not create the VM image, the researcher must search if the VM image is already created (clicking the link ``Search for a VM image'' in the index page of the web system and entering use case~\ref{uc6} --~\nameref{uc6}. If the image is found, the researcher must click on the appropriate link shown in the search results page (the VM image name) which will redirect the researcher to the VM image details' page where a link to launch that VM will be displayed (``Launch VM''), should the VM image be public. If the image is not public, the researcher will have to contact either the VM image creator via the email provided in the search results page or the system administrator.
	\end{itemize}
\item If the researcher clicks the ``Launch VM'' button in the VM image detail's page, he/she will be redirected to the VM launch page, which will display the information regarding the state of the launch. This page will refresh every few seconds and as soon as the VM image is deployed, a warning will be displayed, informing the researcher. The system will also provide the researcher with a username and password combination so that the researcher can access the newly deployed VM.
\end{itemize}

\textbf{Alternative flows}:

The researcher chose to launch a VM after creating it (coming from the use case~\ref{uc3} ---~\nameref{uc3}). In that case the researcher only needs to click the ``Launch VM'' button displayed.

\textbf{Pre-conditions}:

User must be logged in.

\textbf{Post-Conditions}:

If the use case is successful, a VM image will be deployed and will be ready for use.

\subsubsection{UC5 -- View system wide statistics}\label{uc5}

\textbf{Actors}:

\begin{itemize}
\item Researcher/Administrator.
\end{itemize}

\textbf{Brief description}:

A researcher or administrator (known as ``user'' in this use case for simplicity) wishes to see the system statistics.

\textbf{Basic flow}:

\begin{itemize}
\item In order to view the system statistics, the user must be logged in;
\item After a successful login, the user can access the system wide statistics by clicking in the appropriate link ``View statistics'';
\item After being redirected to the statistics page, the user is displayed the following information about the system:
	\begin{itemize}
	\item Most used VM images in the system - shows the user the five most used VM images in the system with a clickable VM image name which redirects to the VM image details;
	\item ``Tag'' cloud - visual representation of the ``tag'' usage in the system i.e. the bigger the font used in the ``tag'' name, the higher the usage. In front of the ``tag'' the VM images which contain that ``tag'' are displayed;
	\item ``Tag'' search frequency - shows the user which ``tags'' were searched more frequently.
	\end{itemize}
\item After viewing the statistics, the user can return to the ``Index'' page by clicking the appropriate link.
\end{itemize}

\textbf{Alternative flows}:

None.

\textbf{Pre-conditions}:

User must be logged in.

\textbf{Post-Conditions}:

None.

\subsubsection{UC6 -- Search for a VM image}\label{uc6}

\textbf{Actors}:

\begin{itemize}
\item Researcher/Administrator.
\end{itemize}

\textbf{Brief description}:

A researcher or administrator (known as ``user'' in this use case for simplicity) wishes to search for a VM image.

\textbf{Basic flow}:

\begin{itemize}
\item In order to search for an existing VM image, the user must be logged in;
\item After a successful login, the user can search for an existing VM by clicking the appropriate link -- ``View statistics'';
\item After being redirected to the search page, the user must input the search terms he/she wishes to search for. The search terms should be separated by commas (``,'');
\item If the system finds any VM images which are tagged with any of the search terms inputted, these VM images will be displayed in a list with clickable names, which redirect to the VM image details' page.
\end{itemize}

\textbf{Alternative flows}:

Search terms are not found. An error message is displayed and the user is redirected to the search page.

\textbf{Pre-conditions}:

User must be logged in.

\textbf{Post-Conditions}:

None.

\subsubsection{UC7 -- View the details of an existing VM image}\label{uc7}

\textbf{Actors}:

\begin{itemize}
\item Researcher/Administrator.
\end{itemize}

\textbf{Brief description}:

A researcher or administrator (known as ``user'' in this use case for simplicity) wishes view the details of an existing VM image. 

\textbf{Basic flow}:

\begin{itemize}
\item In order to view the details of an existing VM image, the user must be logged in;
\item After a successful login,  if the user has created that VM image, he/she must click on his/her username displayed in the top-right corner of the page he/she is currently in and select the desired image in the ``Created images'' section of the page;
\item If the user did not create that VM image, he/she has several options for viewing a VM image detail page:
	\begin{enumerate}
	\item Searching for the VM image using the ``Search'' page and going into the use case~\ref{uc6} --~\nameref{uc6};
	\item Going into use case~\ref{uc5} --~\nameref{uc5} -- and click the VM image names displayed in that page;
	\item Clicking the link ``Launch an existing VM image'' in the ``Index'' page and click the names of the VM images displayed in that page.
	\end{enumerate}
\end{itemize}

\textbf{Alternative flows}:

None.

\textbf{Pre-conditions}:

User must be logged in.

\textbf{Post-Conditions}:

None.

\subsubsection{UC8 -- Modify the details of an existing VM image}\label{uc8}

\textbf{Actors}:

\begin{itemize}
\item Researcher.
\end{itemize}

\textbf{Brief description}:

A researcher wishes to modify the details of an existing VM image. 

\textbf{Basic flow}:

\begin{itemize}
\item In order to modify the details of an existing VM image, the researcher must be logged in and must be the one who created the VM image in question;
\item After a successful login, the researcher must go into use case~\ref{uc7} --~\nameref{uc8}-- and click on his/her username displayed in the top-right corner of the page he/she is currently in and select the desired image in the ``Created images'' section of the page;
\item The researcher must then click the ``Modify VM image'' button and will be redirected to the VM image modification page where he/she will be presented with a form with the details he/she is able to modify;
\item After the user is done updating the desired fields, he/she must click the ``Update'' button.
\end{itemize}

\textbf{Alternative flows}:

The selected VM image is currently being used by another user. The user will not be able to modify the VM image and the use case will fail.

\textbf{Pre-conditions}:

User must be logged in and he/she must be the creator of the VM image.

\textbf{Post-Conditions}:

VM image details are updated.


\subsubsection{UC9 -- View user details}\label{uc9}

\textbf{Actors}:

\begin{itemize}
\item Researcher.
\end{itemize}

\textbf{Brief description}:

A researcher wishes to view his/her own details. 

\textbf{Basic flow}:

\begin{itemize}
\item In order to view his/her own details, the researcher must be logged in;
\item After a successful login, the researcher must click on his/her name displayed in the top-right corner of any page he/she is in;
\item The researcher will then be redirected to the user details' page.
\end{itemize}

\textbf{Alternative flows}:

None.

\textbf{Pre-conditions}:

User must be logged in.

\textbf{Post-Conditions}:

None.

\subsubsection{UC10 -- Modify the user's details}\label{uc10}

\textbf{Actors}:

\begin{itemize}
\item Researcher.
\end{itemize}

\textbf{Brief description}:

A researcher wishes to modify his/her own details. 

\textbf{Basic flow}:

\begin{itemize}
\item In order to modify his/her own details, the researcher must be logged in;
\item After a successful login, the researcher must click on his/her name displayed in the top-right corner of any page he/she is in;
\item The researcher will then be redirected to the user details' page;
\item The researcher must then click on the ``Modify'' button;
\item The researcher will be presented with the ``Modify user details'' page which contains a form with the details which are possible to be modified;
\item After the researcher performs the wanted modifications, he/she must click the ``Update'' button.
\end{itemize}

\textbf{Alternative flows}:

None.

\textbf{Pre-conditions}:

User must be logged in.

\textbf{Post-Conditions}:

User details are updated.




%As mentioned in the above section, one of the main objectives of this dissertation is the integration of an \textit{OpenStack} --- presented in Chapter~\ref{chap:sota} --- deployment environment as it should simplify the cloud creation and management. 


\subsection{The chosen technologies}\label{subsec:tech}

One thing was missing in the cloud computing scene... A cloud management layer. A cloud operating system that added automation and control at scale. That is where \textit{OpenStack} comes into play. As mentioned in Chapter~\ref{chap:sota}, section~\ref{subsec:openstack} it is built by a world wide community of developers, something that made it a good choice to investigate, as the open source culture is something always worthy of enriching.~\cite{stackgithub}

There is one thing one must keep in mind: as it was described in Chapter~\ref{chap:sota}, \textit{OpenStack} is not the only solution available. \textit{OpenNebula} was also available and is already up and running at FEUP. So why choose \textit{OpenStack}?

First of all, \textit{OpenStack} is a more recent project and \textit{OpenNebula}. It is backed up by some renowned names in the industry, such as \textit{Dell}, \textit{AMD}, \textit{Intel}, \textit{Canonical}, \textit{Cisco}, \textit{StackOps}, \textit{HP}, \textit{NEC}, \textit{AT \& T}, \textit{Yahoo!} and \textit{Red Hat}. Some of these companies also support \textit{OpenNebula}.

The coding activity on both projects was also taken into account when chosing which to deploy. With the help of OHLOH~\footnote{An open source directory that anyone can edit. It features comprehensive metrics and analysis on thousands of open source projects.~\cite{ohloh}}, the differences can be easily observed as it is shown in Figure~\ref{fig:ohloh_compare}, which is included in Appendix~~\ref{chap:ap2}.

\textit{OpenStack} has more favourable statistics, such as the number of committers and number of commits (shown in Figures~\ref{fig:committers} and~\ref{fig:commits}) over time. If this is viewed with the knowledge that \textit{OpenNebula} was created first, and \textit{OpenStack} managed to outdo it, great things can be expected.

\begin{figure}[h]
  \begin{center} 
    \leavevmode 
    \includegraphics[scale=0.65]{committers}
    \caption{Comparison between the number of committers on \textit{OpenStack} and \textit{OpenNebula}.~\cite{ohloh}} 
    \label{fig:committers} 
  \end{center}
\end{figure}

\begin{figure}[h]
  \begin{center}
    \leavevmode
    \includegraphics[width=\textwidth]{commits}
    \caption{Comparison between the number of commits on \textit{OpenStack} and \textit{OpenNebula}.~\cite{ohloh}}
    \label{fig:commits}
  \end{center}
\end{figure}

In addition to this and as it was refered in Chapter~\ref{chap:sota}, section~\ref{subsec:opennebula}, \textit{OpenNebula}'s creators have founded an enterprise of their own (C12G Labs) which offers an enterpreise version of \textit{OpenNebula} (named \textit{OpenNebulaPro}). This deviates from the open source philosophy, something that \textit{OpenStack} maintains.

On a more technical aspect, \textit{OpenStack} is mainly written in \textit{Python} whereas \textit{OpenNebula} is mainly coded in \textit{C++} and \textit{Ruby}, as it can be observed in Figure~\ref{fig:code-stack-nebula}. 

In order to understand the relevance of this detail, it must be said that there is an extensive and obligatory contact with \textit{C++} in MIEIC, something that does not happen with \textit{Ruby}, and \textit{Python} is not presented at all. Previous experience with both \textit{C++} and \textit{Ruby} proved to be unfulfilling (\textit{C++} due to its not-so-high-level nature and \textit{Ruby} because it was used in the context of \textit{Ruby on Rails}, which due to some installation quirks was deemed impossible to start and use). 

\textit{Python} on the other hand, was a different programming language, something that was going to be a challenge. Coupled with \textit{Django}, it promised the same advantages as \textit{Ruby on Rails}, but with less trouble getting it up and running. Since \textit{Horizon} is built on \textit{Django}, the choice seemed obvious. The cherry on top of the cake would be contributing to FEUP's knowledge on the new technologies to be researched (\textit{Django}, \textit{Python} and of course, \textit{OpenStack}).

The previous experience with \textit{Ruby} paired with the desire for new challenges and learning new programming languages, \textit{OpenStack} was chosen. This would also allow to contribute to FEUP's knowledge on this new technology.

\begin{figure}[h!]
  \begin{center}
    \leavevmode
    \fbox{\includegraphics[scale=0.5]{code-stack-nebula}}
    \caption{Comparison between the programming languages in \textit{OpenStack} and \textit{OpenNebula}.~\cite{ohloh}}
    \label{fig:code-stack-nebula}
  \end{center}
\end{figure}

Rodrigo Benzaquen, director of site operations and infrastructure at MercadoLibre, a Latin America e-commerce market leader which chose to use \textit{OpenStack} as their cloud solution, stated the following:

\begin{quote}
 ``Before this [\textit{OpenStack}'s deployment], we would have had someone physically deploy the server which would take a day or longer. With \textit{OpenStack}, we don't have to do that; our developers are now able to create and manage their servers.''\cite{openstack-userstories}
\end{quote}

which came directly into the objective of this dissertation, easing the cloud creation and management process.

\clearpage
\subsection{Connecting the dots}\label{subsec:architecture}

As mentioned in Chapter~\ref{chap:sota}, section~\ref{subsec:openstack}, \textit{OpenStack} is designed to deliver a massively scalable cloud operating system, each of the components being designed to work together in order to prodive complete IaaS. This integration is facilitated through pulic APIs that each service offers, being available to the cloud's end users.~\cite{ken-pepple:essex-arch}. 

Expanding the diagram shown in Figure~\ref{fig:openstack_sw_diag}, the relationships between the services are shown in Figure~\ref{fig:openstack_services}:

\begin{figure}[h!]
  \begin{center}
    \leavevmode
    \includegraphics[scale=0.5]{nova-concept-int-essex}
    \caption{Relationships between the different \textit{OpenStack} services.~\cite{ken-pepple:essex-arch}}
    \label{fig:openstack_services}
  \end{center}
\end{figure}

The solution proposed for this project links the \textit{OpenStack} Dashboard --- \textit{Horizon} --- with the designed \textit{Web application} developed in \textit{Python} and \textit{Django}, as shown in Figure~\ref{fig:architecture}.

\begin{figure}[t]
  \begin{center}
    \leavevmode
    \includegraphics[scale=0.5]{architecture}
    \caption{Proposed architecture implementation.}
    \label{fig:architecture}
  \end{center}
\end{figure}

As it can be observed, the web application will use \texttt{vmbuilder} and \textit{cloudinit} whenever needed and then passing that information to the \textit{OpenStack Horizon} dashboard, which will communicate with the rest of \textit{OpenStack} services.

Since \texttt{vmbuilder} and \textit{cloudinit} work for different purposes (\texttt{vmbuilder} creates contextualized VM images and \textit{cloudinit} contextualizes clean VM images), different tools will be used for different purposes.



An interesting feature to complete in future work could be eliminating the web system and passing the image creation and contextualization to \textit{OpenStack}, modifying the \textit{Horizon} dashboard itself.
 

%Este capítulo deve começar por fazer uma apresentação detalhada do
%problema a resolver\footnote{Na introdução a apresentação do
%  problema foi breve.} podendo mesmo, caso se justifique,
%constituir-se um capítulo com essa finalidade.

%Deve depois dedicar-se à apresentação da solução sem detalhes de
%implementação. 
%Dependendo do trabalho, pode ser uma descrição mais teórica, mais
%``arquitectural'', etc.
%\clearpage
\section{Conclusions}

In this chapter was presented the architecture to be followed in Chapter~\ref{chap:chap4} in the implementation phase of the project. The objectives were also outlined, as well as which technologies to use.
